\documentclass{article}
\usepackage{graphicx}
\usepackage{listings}
\usepackage{xcolor}
\usepackage[a4paper, left=1in, right=1in, top=1in, bottom=1in]{geometry}  % Adjust margins here

\lstset{
    basicstyle=\ttfamily\footnotesize,
    keywordstyle=\color{blue},
    commentstyle=\color{green!70!black},
    stringstyle=\color{red},
    breaklines=true,
    frame=single,
    numbers=left,
    numberstyle=\tiny\color{gray},
    tabsize=4,
    showstringspaces=false,
    captionpos=b
}

\title{TCP File Transfer Protocol}
\author{Distributed Systems Practical}
\date{}

\begin{document}

\maketitle

\section{Protocol Design}
Our TCP-based file transfer protocol is designed to allow clients to:
\begin{itemize}
    \item List available files on the server.
    \item Request specific files for download.
    \item Download files reliably using TCP.
    \item Exit the connection gracefully, signaling the server to shut down if needed.
\end{itemize}

\clearpage
\subsection{}{Protocol Design Diagram}

\begin{verbatim}
+---------------------------+                          +---------------------------+
|          Client           |                          |          Server           |
+---------------------------+                          +---------------------------+
| 1. Create socket          |                          | 1. Create socket          |
|                           |                          | 2. Bind socket to address |
|                           |                          | 3. Start listening        |
|---------------------------|                          |---------------------------|
| 2. Connect to server      | -----------------------> | 4. Accept connection      |
|                           | Connection Established   |                           |
|                           | <----------------------- |                           |
|---------------------------|                          |---------------------------|
| 3. Receive file list      | <----------------------- | 5. Send file list         |
| (Display files to user)   |                          |                           |
|---------------------------|                          |---------------------------|
| 4. Request file name      | -----------------------> | 6. Check file existence   |
|                           |                          | If exists:                |
|                           |                          |    - Respond "OK"         |
|                           |                          |    - Transfer file chunks |
|                           |                          | Else:                     |
|                           |                          |    - Respond "File Not    |
|                           |                          |       Found"              |
| Receive file or error     | <----------------------- |                           |
|---------------------------|                          |---------------------------|
| 5. Save file locally      |                          |                           |
|                           |                          |                           |
|---------------------------|                          |---------------------------|
| 6. Send "exit" command    | -----------------------> | 7. Close client socket    |
| (Terminate session)       |                          |                           |
|                           |                          |                           |
|---------------------------|                          |---------------------------|
| Close client socket       |                          | 8. Optionally shut down   |
| Exit program              |                          | the server if required.   |
+---------------------------+                          +---------------------------+
\end{verbatim}

\section{System Organization}
The system consists of two components:
\begin{enumerate}
    \item \textbf{Server:} Hosts the files and serves client requests. Runs continuously and manages multiple clients using threads.
    \item \textbf{Client:} Connects to the server, lists available files, downloads files, and exits when done.
\end{enumerate}

\clearpage
\subsection{System Organization Diagram}

\begin{verbatim}
+--------------------+       TCP Connection       +--------------------+
|      Client        |--------------------------->|       Server       |
|                    |                            |                    |
|  1. Create Socket  |<---------------------------|  1. Create Socket  |
|  2. Connect        |                            |  2. Bind & Listen  |
|                    |         File List          |                    |
|  3. Display List   |<---------------------------|  3. Send File List |
|  4. Request File   |                            |                    |
|--------------------|         File Data          |                    |
|      File Data     |<---------------------------|  4. Transfer File  |
|  5. Save File      |                            |                    |
|  6. Exit Session   |--------------------------->|  5. Close Socket   |
+--------------------+                            +--------------------+
\end{verbatim}

\section{File Transfer Implementation}
The server and client are implemented using Python's \texttt{socket} library. Code snippets of the implementation are shown below.

\subsection{Server Code}
\begin{lstlisting}[language=Python, caption={Server File Handling}]
def handle_client(client_socket, client_address):
    files = os.listdir(SERVER_DIRECTORY)
    file_list = "\n".join(files)
    client_socket.send(file_list.encode())

    while True:
        requested_file = client_socket.recv(BUFFER_SIZE).decode().strip()
        if requested_file.lower() == 'exit':
            break

        file_path = os.path.join(SERVER_DIRECTORY, requested_file)
        if os.path.exists(file_path):
            client_socket.send("OK".encode())
            with open(file_path, 'rb') as f:
                while chunk := f.read(BUFFER_SIZE):
                    client_socket.send(chunk)
        else:
            client_socket.send("ERROR: File not found.".encode())

    client_socket.close()
\end{lstlisting}

\clearpage
\subsection{Client Code}
\begin{lstlisting}[language=Python, caption={Client File Request}]
def start_client():
    client_socket = socket.socket(socket.AF_INET, socket.SOCK_STREAM)
    client_socket.connect((SERVER_HOST, SERVER_PORT))

    file_list = client_socket.recv(BUFFER_SIZE).decode()
    print("Available files:\n" + file_list)

    while True:
        requested_file = input("Enter the file name to download (or 'exit' to quit): ").strip()
        client_socket.send(requested_file.encode())
        if requested_file.lower() == 'exit':
            break

        response = client_socket.recv(BUFFER_SIZE).decode()
        if response == "OK":
            file_path = os.path.join(CLIENT_DIRECTORY, requested_file)
            with open(file_path, 'wb') as f:
                while chunk := client_socket.recv(BUFFER_SIZE):
                    f.write(chunk)
        else:
            print(response)

    client_socket.close()
\end{lstlisting}

\section{Responsibilities}
\begin{itemize}
    \item \textbf{Server:} Handles client connections, provides the list of files, and sends requested files.
    \item \textbf{Client:} Lists available files, requests files for download, and manages local storage.
\end{itemize}

\end{document}
